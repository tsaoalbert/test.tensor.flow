The grade will be soly based on the 7 homework assignments, one per week.
The starter codes with detailed instructions will be provided.

No midterm and final exam. No project presentation.

Late homework policy: Total 72 extra hours can be used for the late submission. 



Week1: introduction 
Slides/Jupter notebook.

Object-Oriented Programming in Python: Defining Classes
  A Fraction Class
  Inheritance: Logic Gates and Circuits

HW#1:
  Reading Chapter 1.

  A.  Extension of the Fraction Class (see Programming Exercises: 1-9)
  B.  Extension of the Fraction Class (see Programming Exercises: 1-9)
  (Starter programs will be given.)



Week2: Analysis
Slides/Jupter notebook.
topic: 
    1. Big-O, 
    2. Anagram Detection Example, 
    3. Performance of Python Data Structures = Lists, Dictionaries


HW#2:
A. Devise experiment to verify that 
          1. verify that the list index operator is O(1)
          2. verify that get item and set item are O(1) for dictionaries.
          3. compares the performance of the del operator on lists and dictionaries.
  (Starter programs will be given.)

B. Given a list of numbers in random order, 
      1. write an algorithm that works in O(nlog(n)) to find the kth smallest number in the list.
      2. (Optional) write an algorithm that works in O(n) to find the kth smallest number in the list.


1st weekend:
  Review HW


Week3:  Basic Data Structures
Slides/Jupter notebook.
Topic: 
    Implementing Stack/Queue/Deque in Python list
    Example: Infix, Prefix and Postfix Expressions
    Implementing Unordered/Ordered Linked List; 
    Analysis of Unordered/Ordered Linked List;
    

HW#3:

      1. Implement a direct infix evaluator that combines both infix-to-postfix conversion and 
      2. Modify the UnorderedList class as follows:
         * allow duplicates
         * remove method can work correctly on non-existing items 
         * Improve the time of length method to O(1)
         * Implement __str__ method so that lists are displayed the Python way 
         * Implement the remaining operations defined in the UnorderedList ADT 
              (append, index, pop, insert).
      3. Implement the remaining operations defined in the OrderedList ADT.

      4. Use linked lists to implement queue, deque, and deque.

      5. Use doubly linked list to implement a queue that would have an average performance of O(1) 
        for enqueue and dequeue operations.



Week4: Recursion
Slides/Jupter notebook.
Topic:
  Recursion vs Iteration
  Three Laws of Recusion
  Example: Tower of Hanoi, Maze, Dynamic Programming

HW#4:

    * Write both recursive and iterative function to compute the factorial of a number.
        How does the performance of the recursive function compare to that of an iterative version?
    * Write both  recursive and iterative function to compute the Fibonacci sequence. 
        How does the performance of the recursive function compare to that of an iterative version?

    * Modify the recursive tree program to produce a more realistic looking tree with various brnach length/thickness, and braching angles.

    * Use DP algorithm to solve the house robber problem.


Week5: Sorting and Searching
  Slides/Jupter notebook.
  Topic:
      Sequential vs Binary Search
      Hashing
        Hash Functions/Collision Resolution
      Sorting:
        Incremental approach O(n^2): Selection/Insertion/Shell Sort
        Divided-and-Conquer  approach O(nlog n): Merge/Quick Sort


HW#5:
  1. Use random ordered integers to analyze the binary search functions given in the text 
     (recursive and iterative). 

  2. Implement the following for the hash table Map ADT implementation.
    len method (__len__) 
    in method (__contains__) 
    del method using the following for collision resolution
          chaining 
          open addressing 
          quadratic probing 

    3. Implement the mergeSort function without using the slice operator.
  
    4. Using a random number generator, create a list of 500 integers. Perform a benchmark analysis using some of the sorting algorithms from this chapter. What is the difference in execution speed?


Week6: Trees and Tree Algorithms
  Topics:
     Tree Traversal: in-order, pre-order, post-order 
     Priority Queues with Binary Heaps
     Binary Search Trees:
     Balnaced Binary Search Trees: AVL/Red-Black Tree
      
HW#6:
  Extend the buildParseTree function to handle 
        * expressions with no spaces between every character.
        * boolean statements (and, or, and not). 

    Using the buildHeap method, write a sorting function that can sort a list in O(nlogn) time.

    Implement a binary heap as a max heap.

    Using the BinaryHeap class, implement a new class called PriorityQueue that 
      implement the constructor, plus the enqueue and dequeue methods.


Week7: Graphs and Graph Algorithms 
  Topics:
    Graph ADT
    implementation of Adjacency Matrix and List
    BFS: Knights' Tour Analysis
    DFS: 
      topological sorting,
      Strongly connected componetns
    Short Path Problems and Dijkstra's Algorithm
    Minmum Spanning TreeProblems and the Prim's/Kruskal Algorithm

HW#7:
    * Modify the depth first search function to produce 
        a topological sort.
        strongly connected components.
    * Using breadth first search revise the maze program from the recursion chapter 
      to find the shortest path out of a maze.
    * Three Water-Jug problems.















  




    






